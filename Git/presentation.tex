\documentclass{beamer} 

\usepackage{xltxtra}
\usepackage{xgreek}
\usepackage{url}
\usepackage{graphicx}
\usepackage{bbding}
\usepackage{gfsneohellenic}
\usetheme{Darmstadt} 
\usecolortheme{crane}

\setsansfont[Mapping=tex-text]{GFS Neohellenic}

\title{Git}
\author{
Ομάδα Web}
\date{\today}

\begin{document}

\begin{frame}
\begin{minipage}{\hsize}
\centering
\end{minipage}
\titlepage
\end{frame} 

\begin{frame}{Τι θα δούμε}
  \begin{itemize}
   \item Τι είναι ένα VCS (Version Control System);
   \pause
   \item Και τι να το κάνω εγώ αυτό;
   \pause
   \item Με λένε git!
   \pause
   \item Γιατί git;
   \pause
   \item Εντολές με PH>7.
   \pause
   \item Επιπλέον υλικό.
  \end{itemize}
\end{frame}

\begin{frame}{Τι είναι ένα Version Control System}
  \begin{itemize}
    \item Καταγράφει τις αλλαγές σε ένα σύνολο αρχείων
    \item Παρέχει δυνατότητα ανάκλησης προηγούμενων στιγμιοτύπων
    \item Διευκολύνει την διαχείριση κοινόχρηστων αρχείων
  \end{itemize}
\end{frame}

\begin{frame}{VCS.getCategories()}
  \begin{itemize}
    \item Local Version Control Systems
    \item Centralized Version Control Systems
    \begin{itemize}
      \item CVS, Subversion, Perforce
    \end{itemize}
    \item Distributed Version Control Systems
    \begin{itemize}
      \item \textbf{Git}, Mercurial, Bazaar, Darcs
    \end{itemize}
  \end{itemize}
\end{frame}

\begin{frame}{Και τι να το κάνω εγώ αυτό;}
 \begin{center}
    Σας θυμίζει κάτι αυτό;\\
    \includegraphics[scale=0.45]{replace.png}
  \end{center}
\end{frame}

\begin{frame}{Και τι να το κάνω εγώ αυτό;}
 Να το χρησιμοποιήσεις για τις ομαδικές εργασίες σου γιατί: 
  \begin{itemize}
    \item Κρατάς - ελέγχεις εύκολα ιστορικό αλλαγών.
    \pause
    \item Ανακαλείς τα αρχεία - project σε προηγούμενη κατάσταση.
    \pause
    \item Παρέχει μια μορφή backup.
    \pause
    \item Μπορείς να μοιραστείς εύκολα τον κώδικα.
    \pause
    \item Ζητείται συχνά γνώση κάποιου VCS σε εργασιακό περιβάλλον.
  \end{itemize}
\end{frame}

\begin{frame}{Εντολές με PH>7}
\begin{itemize}
 \item git init
 \item git status
 \item git add -A
 \item git commit -m <message>
 \item git push
 \item git pull
 \item git clone
\end{itemize}
\end{frame}

\begin{frame}{Γενικό use case}
\begin{itemize}
  \item git clone <link to project>
  \item cd to project folder
  \item make changes
  \item git add -A
  \item git commit -m <message>
  \item git push origin master
\end{itemize}
\end{frame}

\begin{frame}{Επιπλέον υλικό}
\begin{columns}
  \column{.5\textwidth}
    \begin{figure}
      \includegraphics[scale=0.1]{lifejacket.png} 
    \end{figure}
  \column{.5\textwidth}
    \begin{block}{Links}
      \begin{itemize}
	\item  \href{http://git-scm.com/book}{Ανοικτό Βιβλίο}
	\item \href{http://try.github.io/levels/1/challenges/1}{Tutorial}
	\item \href{http://git-scm.com/book/ch4-2.html}{Στήσιμο git repo σε δικό μας server}
	\item \href{https://netbeans.org/kb/docs/ide/git.html}{Χρήση git με το Netbeans}
      \end{itemize}
    \end{block}
\end{columns}
\end{frame}

\end{document}
